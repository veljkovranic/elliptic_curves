\documentclass[12pt]{article}
\usepackage[margin=1in]{geometry}
\usepackage[all]{xy}


\usepackage{amsmath,amsthm,amssymb,color,latexsym}
\usepackage{geometry}        
\geometry{letterpaper}    
\usepackage{graphicx}
\usepackage{amsfonts} 
\usepackage{mathtools}
\DeclarePairedDelimiter\ceil{\lceil}{\rceil}
\DeclarePairedDelimiter\floor{\lfloor}{\rfloor}
\newtheorem{problem}{Problem}

\newenvironment{solution}[1][\textbf{Solution}]{\textbf{#1.} }{$\square$}

\begin{document}
\noindent Big List of Problems\hfill Problem Set - Week 3 \& 4\\
Veljko Vranić (12/09)

\hrulefill


\begin{problem} \textbf{Week 3 - 4.)} \\ \\
Let $E: y^2 = x^3 + Ax + B$ be an elliptic curve over $\mathbb{Q}$ and let $p$ be a prime. \\

(a) If we reduce the coefficients of the equation of $E$ modulo $p$, we get a curve $E: y^2 = x^3 +\overline{A}x +\overline{B}$ over $\mathbb{F}_p$. (The bar denotes reduction modulo $p$.) Is E necessarily an
elliptic curve over $\mathbb{F}_p$? That is, is the discriminant of $E$ nonzero in $\mathbb{F}_p$? (Hint: the
answer is no; justify why.)  \\

(b) The curve $E: y^2 = x^3 + Ax + B$ over $\mathbb{Q}$ is said to have good reduction at $p$ if the reduced curve $E$ is an elliptic curve over $\mathbb{F}_p$. And $E$ has a bad reduction at $p$ if not. \\
Prove that every elliptic curve over $\mathbb{Q}$ has finitely many primes of bad reduction.
\end{problem}

\begin{solution}\\

(a) For $E/\mathbb{Q}$ to be an elliptic curve, it must be the case that its discriminant is $\Delta_{\mathbb{Q}} \neq 0$. \\ 
The same applies for an elliptic curve over $\mathbb{F}_p$, so we should check if $\Delta_{\mathbb{Q}} \neq 0 \implies \Delta_{\mathbb{F}_p} \neq 0$. \\ \\
Let's take an example of $p = 3$. \\
We know that $\Delta_{\mathbb{Q}} = 4A^3 + 27B^2$ and that $\Delta_{\mathbb{F}_p} = 4\overline{A}^3 + 27 \overline{B}^2$. \\
If we consider $A=3$ and any $B \in \mathbb{Q}$, we can see that $\Delta_{\mathbb{Q}} = 4 * 27 + 27B^2 > 0$. \\
However, for our $p=3$ and $A=3$, we have that  $\Delta_{\mathbb{F}_p} = 4 * 27 + 27B^2 = 0 \mod 3$. \\
This counter-example is enough to disprove the proposed implication that we wanted to check. \\ 

(b) Let $E: y^2 = x^3 + Ax + B$ over $\mathbb{Q}$ be some elliptic curve. As we know, $\Delta_{\mathbb{Q}} = 4A^3+27B^2$ is discriminant of the elliptic curve $E$ and it is true that $\Delta_{\mathbb{Q}} \neq 0$. 
Let's consider prime $p$, such that $p \mid \Delta_{\mathbb{Q}}$. We can see that for a reduced curve $E$ over $\mathbb{F}_p$, it holds that \\ $\Delta_{\mathbb{F}_p} \equiv 0 \mod p \iff p \mid \Delta_{\mathbb{Q}}$. As such, $E$ has a bad reduction at $p$. \\

As the set of numbers that divide $\Delta_{\mathbb{Q}}$ is finite, every elliptic curve $E$ has finitely many bad reductions.

\end{solution} 

\begin{problem} \textbf{Week 4 - 2.)} \\ 

There is no obvious analogue for the index calculus approach for the DLP in $E(\mathbb{F}_p)$. Why not? What step fails when you try to generalize it for $E(\mathbb{F}_p)$?
\end{problem}
\begin{solution} \\
In the index calculus algorithm, the critical part is choosing a factor base, a set of $prime$ numbers, which are used in later stages while creating a system of equations (modulo $p$).

The mere notion of $prime$ number is not defined over a $E(\mathbb{F}_p)$, so such an analogue is impossible to trivially generalize. 
\end{solution}
\end{document}
