\documentclass[12pt]{article}
\usepackage[margin=1in]{geometry}
\usepackage[all]{xy}


\usepackage{amsmath,amsthm,amssymb,color,latexsym}
\usepackage{geometry}        
\geometry{letterpaper}    
\usepackage{graphicx}
\usepackage{amsfonts} 
\usepackage{mathtools}
\DeclarePairedDelimiter\ceil{\lceil}{\rceil}
\DeclarePairedDelimiter\floor{\lfloor}{\rfloor}
\newtheorem{problem}{Problem}

\newenvironment{solution}[1][\textbf{Solution}]{\textbf{#1.} }{$\square$}

\begin{document}
\noindent Big List of Problems\hfill Problem Set - Week 1 \& 2\\
Veljko Vranić (11/12)

\hrulefill


\begin{problem} \textbf{Week 1 - 9.)} \\ \\
Let $E : y^2 = x^3 + Ax + B.$\\

(a) Find a polynomial in x whose roots are the x-coordinates of the point \\
$P = (x, y)$ satisfying $3P = \infty$ (Hint. The relation $3P = \infty$ can also be written $2P = - P$.)  \\

(b) For the particular curve $E: y^2 = x^3 + 1$, solve the equation \\
from part (a) to find all points of E satisfying $3P = \infty$. \\
Note that you will need to use complex numbers.
\end{problem}

\begin{solution}\\
The obvious idea following the hint is to express $2P$ using the duplication formula and express it in the equation $2P = -P$.

(a) Points $P$ $x$-coordinate is calculated as following:
$\left( \frac{3x^2_P + A}{2y_P} \right) ^2 - 2x_P$.
We know that point $-P$ is just a reflection against $x$-axis, namely $-P = (x_P, -y_P)$.
That gives us something to work with: 
$$\left( \frac{3x^2_P + A}{2y_P} \right) ^2 - 2x_P = x_P$$.
$$\frac{9x^4_P + 6Ax^2_P + A^2}{4y^2_P} = 3x_P $$.
Assuming that $y_P \neq 0$, we get $$9x^4_P + 6x^2_PA + A^2 = 12 x_P y^2_P $$
Considering the fact that $P$ is a point on an elliptic curve, we know that $y^2_P = x_P^3 + Ax_P+B$.

That gives us:
$$9x^4_P + 6Ax^2_P + A^2 = 12 x_P (x_P^3 + Ax_P+B)$$
$$9x^4_P + 6Ax^2_P + A^2 = 12 (x_P^4 + Ax^2_P+Bx_P)$$
$$-3x^4_P - 6Ax^2_P - 12Bx_P + A^2 = 0$$ or if we multiply by $-1$ to have a bit more positive parameters :) 
$$3x^4_P + 6Ax^2_P + 12Bx_P - A^2 = 0$$.

This polynomial $f(x) = 3x^4 + 6Ax^2 + 12Bx - A^2$ is the one whose roots satisfy the $3P=\infty$.\\

(b) Given $y^2 = x^3 + 1$, we see that $A=0, B=1$. That simplifies our polynomial into $$f(x) = 3x^4+12x$$. Starting our hunt for zeroes, we can clearly divide the whole equation by $3$, and perform some minor grouping
$$x(x^3+4) = 0$$
Obviously, $$x_1=0$$ is one solution and gives us the first two points such that $3P=\infty$, point $P_1=(0,1)$, and $P_2=(0,-1)$. \\
Equation $x^3 = -4$ has three distinct roots (the real and two complex).
Real one is $$x_2=-\sqrt[3]{4}$$
while the complex ones can be written like $$x_{3,4} = \sqrt[3]{4} \left(\frac{1}{2} \pm \frac{\sqrt{3}}{2} i \right)$$.

All these values give us two values for $y$, since from $x^3 + 4 = 0$, we can figure out that $x^3 + 1 = -3$, which means that our $y$ values are $\pm \sqrt{3}$.
This adds up six more points, namely:
$$P_{3,4}=\left(-\sqrt[3]{4},\pm \sqrt{3}i\right), P_{5,6}=\left(\sqrt[3]{4} \left(\frac{1}{2} + \frac{\sqrt{3}}{2}i\right),\pm \sqrt{3}i\right), P_{7,8} = \left(\sqrt[3]{4} \left(\frac{1}{2} - \frac{\sqrt{3}}{2}i\right),\pm \sqrt{3}i\right)$$
The last remaining point is the 'point at infinity', which we can mark by index $P_9$.
\end{solution} 

\begin{problem} \textbf{Week 2 - 7.)} \\ \\
Let $E/\mathbb{Q}$ be an elliptic curve. Prove that $E(\mathbb{Q})_{tors}$ is finite.
\end{problem}
\begin{solution}
Let $E: y^2 = x^3 + Ax + B$ be some elliptic curve over $\mathbb{Q}$. \\
From the Nagell-Lutz theorem, we know that if a point $P \in E(\mathbb{Q})_{tors}$ then two things are known about its coordinates: \\

1.) $x, y \in \mathbb{Z}$ \\

2.) If $y \neq 0$, then $y^2 | \Delta$, where $\Delta = 4A^3 + 27B^2$. \\

As $\Delta$ is some element of $\mathbb{Z}$ it can have a finite set of divisors. Even more precisely, by the fundamental theorem of arithmetic, we can write $\Delta = \prod_i p_i^{q_i}$, where $p_i \in \mathbb{P}$, $\mathbb{P}$ being the set of all prime numbers, and $q_i \in \mathbb{N_0}$.
The exact count of options that $y^2$ can be is $\prod_i \ceil*{\frac{q_i}{2}}$ (each $p_i$ can be ${0, 2, .., \floor*{\frac{q_i}{2}}}$, and independently we can choose exponents for other primes.).
That means that the set of $y$ coordinates is finite. Each of the values of $y$ can give no more than $3$ different values for $x$ since our elliptic curve becomes just a simple cubic equation. This implies that whole $E/\mathbb{Q}$ cannot be infinite.
\end{solution}
\end{document}
